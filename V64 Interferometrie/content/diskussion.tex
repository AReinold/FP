\section{Diskussion}
Für die generelle Betrachtung und als Kommentar zur Durchführung des Versuchs ist anzumerken, dass
versucht worden ist, den Aufbau so exakt wie möglich zu justieren.
Da kleine Verrückungen der Spiegel und des PBSCs schon große Auswirkungen auf den Strahlengang hatten und somit mit
Intensitätsverlusten beim Erfassen mit den Photodioden zu rechnen ist.
Die Kontrastmessung stimmt im Groben mit der Erwartungskurve überein.
Leichte Abweichungen werden auf den oben genannten Grund zurückgefolgert.
Trotzalledem kann angenommen werden, dass das ermittelte Kontrastmaximum als gute Einstellung für die weiteren Messungen dient.\\
Die Abweichung bei der Messung für die Brechungszahl von Glas liegt mit einer Abweichung von $7,4 \%$ zum Theoriewert noch innerhalb der Messungenauigkeit.
Da nicht bekannt ist, um welche Art von Glas es sich handelt, sind eventuelle Abweichungen darauf zurückzuführen.
Der gewählte Theoriewert dient dennoch als gute Referenz für die meisten Glasarten.\\
Für die Gasmessung liegt eine noch geringere Abweichung von $0,001\%$ vor.
Je nach tagesabhängigem Luftdruck kann diese sehr geringe Diskrepanz daher resultieren.
Auch stieg zu Anfang beim Belüften der Gaskammer der Druck kurzzeitig sehr schnell an und es wurde beobachtet, dass das Zählwerk eingige Impulse nicht aufgenommen hat.
Hier hätte bei der Messung genauer gearbeitet werden müssen.
