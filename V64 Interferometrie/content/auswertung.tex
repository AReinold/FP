\section{Auswertung}
Die Auswertung, genauer die Fehlerrechnung, die Plots und Ausgleichsrechnung erfolgt mit den Paketen
Numpy \cite{numpy}, Uncertainties \cite{uncertainties}, Matplotlib \cite{matplotlib} und Scipy \cite{scipy} in der Programmiersprache python.
\subsection{Fehlerrechnung}

\subsection{Kontrastmessung}

\begin{table}[H]
\centering
\begin{tabular}{c|c|c|c}
{$\phi \:/\: \textdegree$} & {$U_\text{max} \:/\: \si{mV}$} & {$U_\text{min} \:/\: \si{mV}$} & {$K$}\\
\midrule
0 & 1394 & 1181 & 0,083 \\
10 & 1194 & 769 & 0,217 \\
20 & 869 & 319 & 0,463 \\
30 & 744 & 175 & 0,619 \\
40 & 781 & 94 & 0,785 \\
50 & 913 & 56 & 0,884 \\
60 & 744 & 90 & 0,784 \\
70 & 919 & 131 & 0,750 \\
80 & 1147 & 469 & 0,420 \\
90 & 963 & 650 & 0,194 \\
110 & 2250 & 344 & 0,735 \\
130 & 1250 & 237 & 0,680 \\
150 & 2156 & 375 & 0,704 \\
170 & 1594 & 813 & 0,324 \\
190 & 1013 & 619 & 0,241 \\
210 & 819 & 188 & 0,627 \\
230 & 1125 & 75 & 0,875 \\
250 & 1297 & 187 & 0,748 \\
270 & 1138 & 788 & 0,182 \\
290 & 2172 & 641 & 0,544 \\
310 & 3562 & 125 & 0,932 \\
330 & 1781 & 375 & 0,652 \\
350 & 1750 & 984 & 0,280 \\
\end{tabular}
\caption{Messwerte der Spannungsextrema der Kontrastmessung.}
\label{tab:kontrast}
\end{table}

\subsection{Brechungsindex der Glasplatten}


\subsection{Brechnungsindex von Luft}
