\section{Aufbau}
Der genutzte Versuchsaufbau ist in Abb.$\ref{aufbau}$ dargestellt. Er besteht aus der, mit einer Heizwicklung umfasseten, Kupferprobe, welche sich in einem Kupferzylinder befindet. Dieser verfügt ebenfalls über eine
separate Heizwicklung. Sowohl der Zylinder, als auch die Probe sind jeweils mit einem Pt-100 Wiederstand ausgestattet, welche als Thermometer dienen. Der Zylinder wird in einem Rezipienten befestigt, welcher an eine
Vakkumpumpe, sowie eine Heliumquelle angeschlossen ist. Der Rezipient wird in ein mit flüssigem Stickstoff gefülltes Dewar-Gefäß gestellt.
