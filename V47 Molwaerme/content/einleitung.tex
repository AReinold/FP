\section{Einleitung}
Das Ziel des Versuchs ist die Bestimmung der Molwärme und der Debye-Temperatur $\Theta_D$ einer Metallprobe. Hierzu ist es zunächst notwendig sich die grundlegenden
Modelle zur Erklärung der Temperaturabhängigkeit der Molwärme zu vergegenwertigen. Bei diesen Modellen handelt es sich um das klassische, das Einstein- und das Debye-Modell.
Zur untersuchung der Probe wird ein Aufbau erarbeitet, welcher es erlaubt die Temperaturabhängigkeit der Molwärme zu bestimmen.
