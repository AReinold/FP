\section{Diskussion}
Werden die beiden berechneten Werte für $\theta_D$ mit dem Literaturwert von $\SI{347}{K}$ verglichen, so fällt eine Abweichung von $48\,\%$ für die experimentell bestimmte Debye-Temperatur und
eine Abweichung von $5\,\%$ für den theoretisch berechneten Wert auf.
Somit zeigt sich, dass die theoretische Vorgehensweise nahe an den Literaturwert heranreicht.
Die mithilfe des Versuchsaufbaus bestimmte Debye-Temperatur weicht hingegen stark vom Literaturwert ab.
Bedingt kann dies einerseits durch nicht ausreichendes Nachfüllen des Stickstoffs sein, da dieser schnell verdampfte und die Probe somit nicht hinreichend gekühlt worden sein könnte,
andererseits konnte schon während der Messung beobachtet werden, dass die Widerstandsdifferenz zwischen Kupferzylinder und -würfel zwischenzeitlich groß gewesen ist, sodass Wärmeverluste nicht ausreichend
verhindert werden konnten.
Auch sollte beachtet werden, dass Restmoleküle trotz des Abpumpens mit der Vakuumpumpe im Rezipienten verblieben sind, die den Wärmetransport förderten und somit Teile der Wärme verloren gegangen sein könnten.
Des Weiteren wurde die Spannungsregelung per Hand getätigt, was dazu führte, dass nicht genau auf das träge Verhalten des, im Vergleich zum Würfel, großen Zylinders eingegangen werden konnte.
