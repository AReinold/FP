\section{Theorie}
\subsection{Das klassische Modell}
Das klassische Modell liefert keinerlei Erklärung für die Temperaturabhängigkeit der Molwärme. Gemäß der klassischen Physik, besagt das Äquipartitionstheorem, dass
sich die thermische Energie in einem Kristall gleichmäßig auf alle Freiheitsgrade der Atome verteilt. Die mittlere Energie pro Freiheitsgrad beträgt hierbei $\frac{1}{2}kT$. Da
es durch die Kristallstruktur den Atomen nur gestattet ist, sich in drei Richtungen um die Glechgewichtslage zu bewegen, folgt für mittlere Energie pro Atom:
\begin{equation}
  \langle u \rangle = 6 \frac{1}{2}\text{k}T\, ,
\end{equation}
wobei k die Boltzmannsche Konstante darstellt. Hochgerechnet auf ein Mol ergibt sich:
\begin{equation}
  U = 3 \text{R}T \, ,
\end{equation}
mit der allgemeinen Gaskonstante R. Gemäß diesem Ergebnis muss die spezifische Molwärme also konstant:
\begin{equation}
  C_v= \left(\frac{\partial U}{\partial T}\right)_V=3R\, .
\end{equation}
Dieser Wert wird zwar für hohe Temperatur erreicht, jedoch zeigt sich, dass die klassische Physik an der Erklärung der Temperaturabhängigkeit der klassischen Physik scheitert
und alternative Modelle notwendig sind.
