\section{Theorie}
\subsection{Das klassische Modell}
Das klassische Modell liefert keinerlei Erklärung für die Temperaturabhängigkeit der Molwärme. Gemäß der klassischen Physik besagt das Äquipartitionstheorem, dass
sich die thermische Energie in einem Kristall gleichmäßig auf alle Freiheitsgrade der Atome verteilt. Die mittlere Energie pro Freiheitsgrad beträgt hierbei $\frac{1}{2}kT$. Da
es durch die Kristallstruktur den Atomen nur gestattet ist, sich in drei Richtungen um die Gleichgewichtslage zu bewegen, folgt für die mittlere Energie pro Atom
\begin{equation}
  \langle u \rangle = 6 \frac{1}{2}\text{k}T\, ,
\end{equation}
wobei k die Boltzmannsche Konstante darstellt. Hochgerechnet auf ein Mol ergibt sich
\begin{equation}
  U = 3 \text{R}T \, ,
\end{equation}
mit der allgemeinen Gaskonstante R. Gemäß dieses Ergebnisses folgt bei einem konstanten Volumen $V$
\begin{equation}
  C_v= \left(\frac{\partial U}{\partial T}\right)_V=3R\, .
\end{equation}
Dieser Wert wird zwar für hohe Temperatur erreicht, jedoch zeigt sich, dass die klassische Physik an der Erklärung der Temperaturabhängigkeit scheitert
und alternative Modelle notwendig sind.
\subsection{Das Einstein-Modell}
Im Einstein-Modell wird angenommen, dass alle Atome auf Gitterplätzen mit der Frequenz $\omega$ schwingen und die Energie, die sie aufnehmen oder abgeben können in ganze Vielfache
von $\symup{\hbar\omega}$ gequantelt ist. Die mittlere Energie pro Oszillator lässt sich über die Boltzmann-Verteilung
\begin{equation}
   W(n) = \exp\left( -\frac{n\hbar\omega}{k_\text{B}T} \right)
\end{equation}
bestimmen. Mit dieser lässt sich die Wahrscheinlichkeit beschreiben, dass ein Oszillator bei Temperatur $T$ eine Energie $n$ besitzt, wenn er sich im Gleichgewicht mit seiner Umgebung befindet.
Wird dies über alle Energie summiert ergibt sich
\begin{equation}
  \langle u\rangle_\text{Einstein}=\frac{\hbar\omega}{e^{\sfrac{\hbar\omega}{kT}}-1}.
\end{equation}
Hiermit folgt für die Molwärme
\begin{align}
  C_\text{V, Einstein} &= 3R\frac{\hbar^2\omega^2}{k_\text{B}^2T^2}\\
   &= \frac{\exp(\hbar\omega/k_\text{B}T)}{[\exp(\hbar\omega/k_\text{B}T)-1]^2}.
\end{align}
Es ist zu erkennen, dass dieser Wert für hohe Temperaturen gegen 3R läuft. Da jedoch die Annahme, dass die Schwingungen im Kristall nur eine Frequenz besitzen, eine grobe
Näherung darstellt, fallen auch im Einstein-Modell Abweichungen zu den experimentellen Werten auf.
\subsection{Das Debye-Modell}
Das Debye-Modell liefert eine genauere Näherung an die experimentellen Werte, indem es, anders als das Einstein-Modell, von einer spektralen Verteilung $Z(\omega)$ der Eigenschwingungen ausgeht.
Diese komplizierte Verteilung wird dabei durch die Annahme, dass die Phasengeschwindigkeit von der Frequenz und der Ausbreitungsrichtung der Welle im Kristall unabhängig ist, vereinfacht.
Da ein Kristall aus $N_\text{L}$ Atome nur $3N_\text{L}$ Eigenschwingungen besitzen kann, muss es eine maximale Grenzfrequenz $\omega_\text{D}$ geben, für welche gilt
\begin{equation}
  \int_0^{\omega_\text{D}} Z(\omega)\text{d}\omega = 3N_\text{L}\,.
\end{equation}
