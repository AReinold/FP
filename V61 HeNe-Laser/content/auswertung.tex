\section{Auswertung}
\subsection{Polarisation}
% I_0=2.68 \pm 0.05 µA
%\phi=72.5°+-0.96°
\subsection{Wellenlänge}
% n=1 d_n=6,4cm \lambda=639.2nm
% n=2 d_n=6,4cm+6.3 \lambda=631.7nm
% n=3 d_n=6,4cm+6.3+6.5 \lambda=632.6nm
% n=4 d_n=6,4cm+6.3+6.5+6.5 \lambda=629.3nm
% n=5 d_n=6,4cm+6.3+6.5+6.5+6.4 \lambda=621.8nm
% n=6 d_n=6,4cm+6.3+6.5+6.5+6.4+6.5 \lambda=614.7nm
\subsection{TEM Moden}
%Grundmode I=I_0*exp(-2*((x-d)/w)**2):
%I=1.04443 +- 0.03142
%d=6.79377 +- 0.1143
%w=6.59133 +- 0.2345
%Erste Mode
%I(r)=a\cdot\frac{8(x-b)^2}{c^2}\cdot\symup{e}^{\frac{-2(x-b)^2}{c^2}}
%a=0.235343 +- 0,01453
%b=8,5871 +- 0,1666
%c=5,83715 +- 0,2477
\subsection{Stabilitätsmessung}
%f(x)=a*x^2+bx+c
%a=0.00172 +- 0.0007
%b=-0.389 +- 0,1425
%c=22.134+- 7,027
\subsection{Wellenlängenbestimmung}
Die zur Bestimmung der Wellenlänge $\lambda$ aufgenommenen Werte, sowie die daraus berechneten Wellenlängen sind in Tabelle $\ref{welltab}$ dargestellt. Für $\lambda$ gilt hierbei:
\begin{equation}
  \lambda = \frac{b}{n}\cdot\sin\left(\tan^{-1}\left(\frac{s_n}{L}\right)\right) \, .
\end{equation}
Hierbei bezeichnet n die Ordnung des Maximums, $b=\frac{1}{80}\si{\milli\meter}$ die Gitterbreite, $s_n$ den Abstand des $n$-ten Maximums zum $0$-ten Maximum und $L$ den Abstand zwischen Gitter und Schirm,
welcher hier $L=125 \si{\centi\meter}$ beträgt.
\begin{table}[H]
  \centering
\begin{tabular}{c|c|c}
$n$  & $s_n$ [cm]    & $\lambda$ [nm]     \\
\hline
1 & 6,4 & 639,2 \\
2 & 6,3 & 631,7 \\
3 & 6,5 & 632,6 \\
4 & 6,5 & 629,3 \\
5 & 6.4 & 621,8 \\
6 & 6.5 & 614,7
\end{tabular}
\end{table}
Als Mittelwert ergibt sich hierbei:
\begin{equation}
  \lambda= (628,2 \pm 8,7) \, \si{\nano\meter}
\end{equation}
\subparagraph{Modenuntersuchung}
\subsubsection{Grundmode}
Zur Untersuchung der Grundmode wird 
