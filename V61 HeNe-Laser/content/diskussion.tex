\section{Diskussion}
Die ermittelten Werte weisen zum Teil eine gute Übereinstimmung mit der Theorie auf. So weicht die berechnete Wellenlänge nur um $4,6\,\si{\nano\meter}$ vom Literaturwert
$\lambda=632,8\,\si{\nano\meter}$ \cite{anleitung} ab, was innerhalb der Messungenauigkeit liegt. Die Polarisationsuntersuchung zeigt eine gute Übereinstimmung zwischen
den Messwerten und der Ausgleichsfunktion. Auch die TEM-Moden entsprechen den theoretischen Erwartungen, mit zwei deutlich erkennbaren Maxima für die TEM$_{01}$-Mode.\\
Die Stabilitätsmessung zeigte, dass bei einer Kombination aus zwei konkaven spegeln eine größere Resonatorlänge möglich ist als bei einer Kombination aus planaren und konkaven Spiegeln.
Dies deckt sich mit dem aus der Theorie erwartbaren. Bei der Spiegelkonfiguration mit einem planaren Spiegel fällt jedoch auf, dass die maximale Resonatorlänge unter dem theoretischen Maximum von $140 \, \si{\centi\meter}$ liegt. Eine
wahrscheinliche hierfür Erklärung liegt am Versuchsaufbau, da es beim Verschieben der Spiegel nur schwer möglich ist den Strahl aufrechtzuerhalten.
