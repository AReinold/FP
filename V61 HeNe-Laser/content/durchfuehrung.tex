\section{Versuchsaufbau und -durchführung}
Der Helium-Neon-Laser ist, wie der Name schon sagt, ein Laser mit einem Gasgemisch aus Helium- und Neon-Atomen.
Das Verhältnis beträgt 5:1.
Das Gas befindet sich in einem Laserrohr, an dessen Enden jeweils ein Brewsterfenster ist. Es dient dazu, Reflexionsverluste zu minimieren.
Am Laserrohr wird eine Spannung angelegt, sodass über elektrische Entladungen eine Besetzungsinversion geschieht.
Durch diese Entladungen wird das Helium in metastabile Zustände angeregt und durch Stöße zweiter Art wird die Anregungsenergie weitergegeben, dass Besetzungsinversion auftritt.
Die rote Linie mit $\lambda = \SI{632.8}{nm}$ ist die dominate.
