\section{Versuchsaufbau und -durchführung}
Alle an der Justage und an der anschließenden Messung beteiligten Komponenten sind auf einer optischen Schiene beweglich angeordnet.
Für die Justage ist ein Justagelaser mit einer Wellenlänge von $\lambda = 532\, \si{nm}$, einer maximalen Leistung $P_\text{max}=1\,\si{mW}$ und einer reduzierten Laserleistung $P_\text{red}=0,2\, \si{mW}$ montiert.
Für den eigentlichen HeNe-Laser befinden sich zwei austauschbare hochreflektierende Spiegel an den Enden des Resonators.
Das gesamte Laserrohr hat eine Länge von $l=408\,\si{mm}$ und einen Durchmesser von $d_\text{HeNe}=1,1\,\si{mm}$.
Für das Vermessen der Eigenschaften des Lasers stehen ebenfalls mehrere Komponenten wie eine Photodiode oder Polarisationsfilter zur Verfügung.\\

Vor Beginn der eigentlichen Abschnittsmessungen muss der HeNe-Laser justiert werden.
Erst dann kann mit der Überprüfung der Stabilitätsmessung begonnen werden. Dafür wird der Laserstrahl mithilfe einer Photodiode auf seine maximale Leistung eingestellt.
Im Folgenden wird der Abstand der Laserspiegel immer weiter vergrößert und der Laser nachjustiert. Daraus lässt sich ableiten, ob der theoretisch errechnete Wert auch in der Praxis erreicht werden kann.
Anschließend folgt die Messung der TEM-Moden.
Für die Stabilisierung der Moden wird ein dünner Wolframdraht in den Strahlengang gespannt. Mit einer fein verstellbaren Photodiode werden möglichst viele vermessen.
Ebenso wird eine Polarisationsmessung durchgeführt. Dazu wird ein verstellbarer Polarisationsfilter hinter den Auskoppelspiegel gesetzt und die Strahlenintensität bei verschiedenen Polarisationswinkeln gemessen.\\
Als letztes folgt die Messung zur Bestimmung der Wellenlänge, die aus den Beugungsminima und -maxima eines eingesetzten Gitters berechnet werden kann.
