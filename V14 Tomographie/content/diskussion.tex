\section{Diskussion}
Die Literaturwerte, welche genutzt wurden, um die Komposition von Würfel 5 zu bestimmen, befinden sich in Tabelle $\ref{litwerte}$.
\begin{table}[h!]
  \centering
  \caption{Absorptionskoeffizienten, der im Versuch genutzten Materialien.\cite{koeff}.}
  \begin{tabular}{c
                  S[table-format=1.3]
									S[table-format=2.2]}
    \toprule
    {Material} & {$\mu$, $\si{\per\centi\meter}$}& {$\sigma$, $\si{\centi\meter\squared\per\gram}$} \\
		\midrule
    Blei &1.245& 0.110 \\
    Messing &0.614 & 0.073 \\
	Eisen &0.574& 0.073 \\
	Aluminium & 0.203 & 0.075 \\
	Delrin &0.116& 0.082 \\
    \bottomrule
  \end{tabular}
  \label{litwerte}
\end{table}\\
Wie zu erkennen ist, liegt zwischen den berechneten Absorptionskoeffizienten für Aluminium und Blei und den Literaturwerten nur eine Abweichung von etwa $8,4$\% beziehungsweise $1.2$\%,
was im Rahmen der Messungenauigkeit liegt, sodass nicht von systematischen Fehlern auszugehen ist.\\
Die Abweichungen zwischen den Absorptionskoeffizienten der Teilwürfel und denen der zugeordneten Materialien sind hier größer, wobei die Abweichung bei Würfel 9 am
größten ist. Bei diesem und Teilwürfel 4 fällt außerdem auf, dass der Absorptionskoeffizient negativ ist, sodass hier von einem systematischen Fehler auszugehen ist.\\
Die Untersuchung des leeren Würfels zeigt, dass die Intensitätsvektoren voneinander abweichen. Eine Erklärung hierfür ist, neben Messungenauigkeiten, dass die genutzten
Messapparaturen eine gewisse Zeit benötigen um warm zu laufen. Dies verursacht eine Verfälschung der frühen Messwerte.\\
Eine weitere Fehlerquelle stellt die mangelnde Fokussierung des Strahls dar, wodurch Intensitätsverluste unabhängig von Absorptionskoeffizienten auftreten.
