\section{Theorie}
\subsection{Grundlagen}
Zur Untersuchung mittels Tomographie wird das Objekt, das untersucht werden soll, mit Strahlung durchdrungen.
Bei dieser Strahlung kann es sich, je nach zu untersuchendem Körper, um Teilchenstrahlung oder um radioaktive Strahlung handeln.
Durch die Abschwächung, die der Strahl durch das Durchdringen erfährt, kann über die gemessene Intensität auf die Materialeigenschaft rückgeschlossen werden.
Wenn das Objekt von unterschiedlichen Seiten bestrahlt wird, lässt sich so ein räumliches Bild erstellen.
In diesem Versuch wird als $\gamma$-Strahlungsquelle $\ce{^{137}Cs}$ verwendet. Es zerfällt zu $\ce{^{137}Ba}$.
Mit einer Wahrscheinlichkeit von $6,5 \percent$ zerfällt es direkt in den Grundzustand.
Zum größten Teil aber, zerfällt es in einen angeregten Bariumkern, der unter Aussendung eines Photons schließlich auch in den Grundzustand fällt.
\subsection{Bestimmung des Absorptionskoeffizienten}
