\section{Auswertung}
Die Auswertung, genauer die Fehlerrechnung, die Plots und Ausgleichsrechnung erfolgt mit den Paketen
Numpy \cite{numpy}, Uncertainties \cite{uncertainties}, Matplotlib \cite{matplotlib} und Scipy \cite{scipy} in der Programmiersprache python.
\subsection{Fehlerrechnung}
Zu Beginn der Auswertung sei gesagt, dass eine Ungenauigkeit von $\SI{30}{\percent}$ des analogen zum digitalen Glühkathodenvakuummessgeräts
bei der Messung festgestellt und im Folgenden berücksichtigt wurde.
Auch muss erwähnt werden, dass die Zeitmessung mit einer Handystoppuhr erfolgte und eine durch menschliche Reaktion verursachte
Verzögerung von $\SI{180}{ms}$ beachtet wird.\cite{reaktion}
In den nachfolgenden Messwerttabellen sind die Abweichungen schon miteingerechnet.\\
Die Mittelwerte werden nach
\begin{equation}
	\bar{x}=\frac{1}{N}\sum_{i}^N x_i
\end{equation}
und deren Standardabweichung mit
\begin{equation}
	\sigma_{\bar{x}} = \sqrt{\frac{1}{N(N-1)} \sum_{i}^N (x_i-\bar{x})^2}
\end{equation}
berechnet.
Für die Fehlerfortpflanzung einer Variablen $x_i$ gilt
\begin{equation}
	\sigma = \sqrt{\sum_{i}^N \Bigl(\frac{\partial f}{\partial x_i} \sigma_{x_i}\Bigr)^2}.
	\label{eq:gaussfehler}
\end{equation}
\subsection{Bestimmung des Volumens}
Für die folgende Auswertung ist es unumgänglich, das Volumen des Messaufbaus zu bestimmen.
Tabelle $\ref{tab:Bauteile}$ zeigt, aufgelistet nach Pumpen- und Messart die verwendeten Bauteile und ihre Volumina.
Es wird zwischen der Evakuierungsmessung mit der Turbopumpe TE, derselben Messart mit der Drehschieberpumpe DE und der Leckratenmessung
mit der Drehschieberpumpe DL und mit der Turbopumpe TL unterschieden.\newline

\begin{table}[!hht]
	\centering
	\begin{tabular}{|c|c|c|c|c|c|c|}
		\hline
		Bauteil Nr. & Bauteilbezeichnung & {$\symup{Volumen} \:/\: l$}& TE & TL & DE & DL\\ \hline
		1	&	Tank & $9,5 \pm 0,8$ & 1 & 1 & 1 & 1 \\ \hline
		2	&	langer Schlauch & $0,8 \pm 0,1$ &  &  & 1 & 1 \\	\hline
		3	&	kurzer Schlauch & $0,087 \pm 0,011$ &  &  & 1 & 1 \\	\hline
		4a	&	T-Stück, klein & $0,013 \pm 0,002$ &  &  & 1 & 1 \\	\hline
		4b	&	T-Stück, groß & $0,25 \pm 0,01$ & 1 & 1 & 1 & 1 \\	\hline
		5a	&	Kreuzstück, klein & $0,016 \pm 0,002$ &  &  & 1 & 1 \\	\hline
		5b	&	Kreuzstück, groß & $0,177 \pm 0,09$ & 1 & 1 & 1 & 1 \\	\hline
		6a	&	Handventil 1, offen & $0,015 \pm 0,002$ &  & 1 & 1 & 2 \\	\hline
		6b	&	Handventil 1, geschlossen & $0,005 \pm 0,001$ &  2 & 1 & 2 & 1 \\	\hline
		7a	&	Handventil 2, offen & $0,025 \pm 0,005$ &  &  & 1 &  \\	\hline
		7b	&	Handventil 2, geschlossen & $0,0125 \pm 0,0025$ &  &  &  & 1 \\	\hline
		8a	&	Klappenventil, offen & $0,044 \pm 0,004$ & 1 &  &  &  \\	\hline
		8b	&	Klappenventil, geschlossen & $0,022 \pm 0,002$ &  & 1 & 1 & 1 \\	\hline
		9	&	Querschnittsverengung & $0,067 \pm 0,004$ & 1 &  &  &  \\	\hline
	\end{tabular}
	\caption{Auflistung der Bauteile.\cite{anleitung}}
	\label{tab:Bauteile}
\end{table}
Die Teilvolumina der jeweiligen Aufbauten werden addiert und im Folgenden angegeben. Die Fehlerberechnung erfolgt mit Formel $\ref{eq:gaussfehler}$.
Es ergeben sich
\begin{align*}
   V_\text{TE}=\SI{10,0 \pm 0,8}{l}\\
   V_\text{TL}=\SI{10,0 \pm 0,8}{l}\\
   V_\text{DE}=\SI{10,9 \pm 0,8}{l}\\
   V_\text{DL}=\SI{10,9 \pm 0,8}{l}.
\end{align*}

\subsection{Turbomolekularpumpe}
Für die Turbomolekularpumpe werden im Folgenden beide Messverfahren ausgewertet.
Zum anschließenden Vergleich sind folgende Herstellerangaben gegeben:\\
Turbo SST81 der Firma ILMAC mit einem Saugvermögen von $\SI{77}{l/s}$
\subsubsection{Evakuierungsmessung}
\begin{table}[H]
\tiny
\centering
\label{tab:EvakuierungskurveTurbo}
\begin{tabular}{c|c|c|c|c|c|c|c|c|c|c}
{$p \:/\: \si{mbar}$} & {$\ln{\Bigl( \frac{p(t)-p_E}{p_0-p_E}\Bigr)}$} & {$t_1 \:/\: \si{s} $} & {$t_2 \:/\: \si{s}$} & {$t_3 \:/\: \si{s}$} & {$t_4 \:/\: \si{s}$} & {$t_5 \:/\: \si{s}$} & {$t_6 \:/\: \si{s}$}& {$t_7 \:/\: \si{s}$} & {$t_8 \:/\: \si{s}$} & {$\bar{t} \:/\: \si{s}$}\\
\midrule
$(8,05 \pm \, 0,81)\cdot 10^{-3}$ & 0 & 0 &  0 & 0 & 0 & 0 & 0 & 0 & 0 & 0\\
$(6,0 \pm \, 0,6)\cdot 10^{-3}$ & $-(0,29 \pm \, 0,14)$ & 0,44 & 0,87 & 0,78 & 0,70 & 0,79 & 0,87 & 0,78 & 0,83 & $0,76 \pm \, 0,05$\\
$(3,0 \pm \, 0,3)\cdot 10^{-4}$ & $-(0,98 \pm \, 0,14)$ & 0,74 & 1,40 & 1,21 & 1,09 & 1,35 & 1,43 & 1,41 & 1,62 & $1,28 \pm \, 0,10$\\
$(6,0 \pm \, 0,6)\cdot 10^{-4}$ & $-(2,62 \pm \, 0,14)$ & 2,82 & 3,13 & 2,82 & 3,00 & 3,10 & 3,20 & 3,09 & 2,79 & $3,00 \pm \, 0,06$\\
$(3,0 \pm \, 0,3)\cdot 10^{-5}$ & $-(3,34 \pm \, 0,15)$ & 4,01 & 4,04 & 3,99 & 3,87 & 3,93 & 4,02 & 3,93 & 4,03 & $3,94 \pm \, 0,04$\\
$(6,0 \pm \, 0,6)\cdot 10^{-5}$ & $-(5,20 \pm \, 0,18)$ & 6,52 & 6,86 & 6,61 & 7,04 & 6,44 & 6,72 & 6,71 & 6,70 & $6,60 \pm \, 0,10$\\
$(3,0 \pm \, 0,3)\cdot 10^{-5}$ & $-(6,35 \pm \, 0,28)$ & 10,14 & 10,03 & 10,00 & 9,44 & 10,29 & 9,42 & 9,06 & 8,94 & $9,67 \pm \, 0,20$\\
$(1,6 \pm \, 0,2)\cdot 10^{-5}$ & - & - &  -& -& -& -& -& -& -& - \\
\end{tabular}
\caption{Messwerte für die Evakuierungskurve der Turbomolekularpumpe.}
\end{table}

\begin{figure}[H]
  \centering
  \includegraphics[scale=0.27]{bilder/turbo exponential.png}
  \caption{Exponentielle Evakuierungskurve Turbomolekularpumpe 3.}
  \label{turboexponential}
\end{figure}

\begin{figure}[H]
  \centering
  \includegraphics[scale=0.42]{bilder/turbodruck plot.png}
  \caption{Linearer Plot Evakuierungskurve Turbomolekularpumpe.}
  \label{turboplot}
\end{figure}

\subsubsection{Leckratenmessung}
\begin{table}[H]
\centering
\label{tab:leck_Turbo1}
\begin{tabular}{c|c|c|c|c}
	{$p \:/\: \si{mbar}$} & {$t_1 \:/\: \si{s} $} & {$t_2 \:/\: \si{s} $} & {$t_3 \:/\: \si{s} $} & {$\bar{t} \:/\: \si{s}$}\\
\midrule
$(5,0 \pm \, 0,5)\cdot 10^{-5}$ &0 &0 &0 &0\\
$(8,0 \pm \, 0,8)\cdot 10^{-5}$ &   0,32 &  0,66 &  0,39 & $0,46 \pm 0,08$\\
$(2,0 \pm \, 0,2)\cdot 10^{-4}$ &   1,71  &  1,83 &  1,65 & $1,73 \pm 0,04 $\\
$(4,0 \pm \, 0,4)\cdot 10^{-4}$ &   5,91 &  6,02 &  6,40 & $6,11 \pm 0,12 $\\
$(8,0 \pm \, 0,8)\cdot 10^{-4}$ &   13,05 &  13,14 &  14,02 & $13,40 \pm 0,25 $\\
$(2,0 \pm \, 0,2)\cdot 10^{-3}$ &   34,01 &  34,50 &  36,37 & $35,96 \pm 0,59 $\\
$(4,0 \pm \, 0,4)\cdot 10^{-3}$ &  61,18 & 62,18 & 66,18 & $63,18 \pm 1,25 $\\
\end{tabular}
\caption{Gleichgewichtsdruck bei $p_G=(5,0 \pm \, 0,5)\cdot 10^{-5} \, \si{mbar}$}
\end{table}

\begin{table}[H]
\centering
\label{tab:leck_Turbo2}
\begin{tabular}{c|c|c|c|c}
	{$p \:/\: \si{mbar}$} & {$t_1 \:/\: \si{s} $} & {$t_2 \:/\: \si{s} $} & {$t_3 \:/\: \si{s} $} & {$\bar{t} \:/\: \si{s}$}\\
\midrule
$(10 \pm \, 1)\cdot 10^{-5}$ &0 &0 &0 &0\\
$(3,0 \pm \, 0,3)\cdot 10^{-4}$ &   1,22 &  0,96 &  0,70 & $0,96 \pm 0,06$\\
$(6,0 \pm \, 0,6)\cdot 10^{-4}$ &   3,00  &  2,77 &  2,69 & $2,82 \pm 0,08 $\\
$(9,0 \pm \, 0,9)\cdot 10^{-4}$ &   4,85 &  4,56 &  4,47 & $4,63 \pm 0,10 $\\
$(2,0 \pm \, 0,2)\cdot 10^{-3}$ &   10,88 &  10,78 &  11,30 & $11,00 \pm 0,13 $\\
$(4,0 \pm \, 0,4)\cdot 10^{-3}$ &   20,09 &  20,00 &  19,61 & $19,90 \pm 0,12 $\\
$(6,0 \pm \, 0,6)\cdot 10^{-3}$ &  28,26 & 27,78 & 27,36 & $27,80 \pm 0,21 $\\
\end{tabular}
\caption{Gleichgewichtsdruck bei $p_G=(10 \pm \, 1)\cdot 10^{-5} \, \si{mbar}$}
\end{table}

\begin{table}[H]
\centering
\label{tab:leck_Turbo3}
\begin{tabular}{c|c|c|c|c}
	{$p \:/\: \si{mbar}$} & {$t_1 \:/\: \si{s} $} & {$t_2 \:/\: \si{s} $} & {$t_3 \:/\: \si{s} $} & {$\bar{t} \:/\: \si{s}$}\\
\midrule
$(1,5 \pm \, 0,2)\cdot 10^{-5}$ &0 &0 &0 &0\\
$(3,0 \pm \, 0,3)\cdot 10^{-4}$ &   0,57 &  0,40 &  0,40 & $0,46 \pm 0,05$\\
$(6,0 \pm \, 0,6)\cdot 10^{-4}$ &   1,27  &  1,36 &  1,40 & $1,34 \pm 0,03 $\\
$(9,0 \pm \, 0,9)\cdot 10^{-4}$ &   2,71 &  2,55 &  2,48 & $2,58 \pm 0,06 $\\
$(2,0 \pm \, 0,2)\cdot 10^{-3}$ &   6,70 &  6,31 &  6,22 & $6,41 \pm 0,12 $\\
$(4,0 \pm \, 0,4)\cdot 10^{-3}$ &   12,30 &  11,94 &  11,89 & $12,04 \pm 0,11 $\\
$(6,0 \pm \, 0,6)\cdot 10^{-3}$ &  17,37 & 17,31 & 17,10 & $17,26 \pm 0,07 $\\
\end{tabular}
\caption{Gleichgewichtsdruck bei $p_G=(1,5 \pm \, 0,2)\cdot 10^{-4} \, \si{mbar}$}
\end{table}

\begin{table}[H]
\centering
\label{tab:leck_Turbo4}
\begin{tabular}{c|c|c|c|c}
	{$p \:/\: \si{mbar}$} & {$t_1 \:/\: \si{s} $} & {$t_2 \:/\: \si{s} $} & {$t_3 \:/\: \si{s} $} & {$\bar{t} \:/\: \si{s}$}\\
\midrule
$(2,0 \pm \, 0,2)\cdot 10^{-4}$ &0 &0 &0 &0\\
$(5,0 \pm \, 0,5)\cdot 10^{-4}$ &   0,64 &  0,82 &  0,74 & $0,73 \pm 0,04$\\
$(8,0 \pm \, 0,8)\cdot 10^{-4}$ &   1,57  &  1,83 &  1,65 & $1,68 \pm 0,06 $\\
$(2,0 \pm \, 0,2)\cdot 10^{-3}$ &   4,76 &  4,69 &  5,04 & $4,83 \pm 0,09 $\\
$(4,0 \pm \, 0,4)\cdot 10^{-3}$ &   9,57 &  9,77 &  9,71 & $9,68 \pm 0,05 $\\
$(6,0 \pm \, 0,6)\cdot 10^{-3}$ &   13,67 &  13,94 &  13,78 & $13,80 \pm 0,06 $\\
$(8,0 \pm \, 0,8)\cdot 10^{-3}$ &  17,50 & 17,87 & 17,59 & $17,65 \pm 0,09 $\\
\end{tabular}
\caption{Gleichgewichtsdruck bei $p_G=(2,0 \pm \, 0,2)\cdot 10^{-4} \, \si{mbar}$}
\end{table}

\begin{figure}[H]
  \centering
  \includegraphics[scale=0.36]{bilder/leckratefit1.png}
  \caption{Leckrate Drehschieberpumpe 1.}
  \label{leckrate1}
\end{figure}
\begin{figure}[H]
  \centering
  \includegraphics[scale=0.40]{bilder/leckratefit2.png}
  \caption{Leckrate Drehschieberpumpe 2.}
  \label{leckrate2}
\end{figure}
\begin{figure}[H]
  \centering
  \includegraphics[scale=0.42]{bilder/leckratefit3.png}
  \caption{Leckrate Drehschieberpumpe 3.}
  \label{leckrate3}
\end{figure}
\begin{figure}[H]
  \centering
  \includegraphics[scale=0.38]{bilder/leckratefit4.png}
  \caption{Leckrate Drehschieberpumpe 4.}
  \label{leckrate4}
\end{figure}

\subsection{Drehschieberpumpe}
Ebenso für die Drehschieberpumpe folgt die Auswertung mit anschließendem Vergleich mit der
Herstellerangabe:\\
Drehschieber Pfeiffer Duo 004A mit einem Saugvermögen von $\SI{1,1}{l/s}$\\
Die folgende Tabelle zeigt die Messwerte zur Bestimmung der Evakuierungskurve.
\subsubsection{Evakuierungsmessung}
\begin{table}[H]
\centering
\label{tab:EvakuierungskurveDreh}
\begin{tabular}{c|c|c|c|c|c|c|c}
{$p \:/\: \si{mbar}$} & {$\ln{\Bigl( \frac{p(t)-p_E}{p_0-p_E}\Bigr)}$} & {$t_1 \:/\: \si{s} $} & {$t_2 \:/\: \si{s}$} & {$t_3 \:/\: \si{s}$} & {$t_4 \:/\: \si{s}$} & {$t_5 \:/\: \si{s}$} & {$\bar{t} \:/\: \si{s}$}\\
\midrule
$(1000 \pm \, 300)$ & 1 & 0 &  0 & 0 & 0 & 0 & 0\\
$(100 \pm \, 30)$ & -(2,3 \pm \, 0,4) & 18,62 & 11,59 & 16,47 & 15,02 & 16,43 & $15,63 \pm \, 1,04$\\
$(40 \pm \, 12)$ & -(3,2 \pm \, 0,4) & 35,33 & 26,49 & 32,97 & 34,34 & 33,50 & $32,53 \pm \, 1,40$\\
$(10 \pm \, 3)$ & -(4,6 \pm \, 0,4) & 49,85 & 42,19 & 49,28 & 48,84 & 49,01 & $47,83 \pm \, 1,27$\\
$(6 \pm \, 1,8)$ & -(5,1 \pm \, 0,4) & 54,99 & 46,42 & 54,25 & 54,31 & 54,18 & $52,83 \pm \, 1,44$\\
$(4 \pm \, 1,2)$ & -(5,5 \pm \, 0,4) & 59,42 & 50,29 & 57,74 & 57,65 & 58,36 & $56,69 \pm \, 1,46$\\
$(2 \pm \, 0,6)$ & -(6,2 \pm \, 0,4) & 65,83 & 57,38 & 64,60 & 64,50 & 64,79 & $63,42 \pm \, 1,37$\\
$(1 \pm \, 0,3)$ & -(6,9 \pm \, 0,4) & 73,00 & 64,02 & 71,62 & 69,82 & 71,97 & $70,09 \pm \, 1,43$\\
$(0,6 \pm \, 0,18)$ & -(7,5 \pm \, 0,4) & 79,54 & 70,73 & 77,92 & 77,80 & 78,18 & $76,8 \pm \, 1,43$\\
$(0,4 \pm \, 0,12)$ & -(7,9 \pm \, 0,4) & 86,35 & 77,33 & 85,30 & 85,24 & 85,19 & $83,9 \pm \, 1,54$\\
$(0,2 \pm \, 0,06)$ & -(8,7 \pm \, 0,5) & 99,69 & 90,02 & 98,98 & 98,17 & 98,83 & $97,14 \pm \, 1,61$\\
$(0,1 \pm \, 0,03)$ & -(9,7 \pm \, 0,6) & 111,79 & 101,65 & 110,30 & 111,17 & 111,87 & $109,36 \pm \, 1,74$\\
$(0,06 \pm \, 0,02)$ & -(10,8 \pm \, 1,2) & 125,13 & 113,88 & 128,72 & 129,62 & 130,60 & $125,59 \pm \, 2,75$\\
$(0,04 \pm \, 0,01)$ & - & - & - & - & - & - & -
\end{tabular}
\caption{Messwerte für die Evakuierungskurve der Drehschieberpumpe.}
\end{table}



Je nach Druckbereich, wird der Graph unterteilt und gesonderte lineare Fits erstellt.
Es folgt die graphische Darstellung der logarithmierten Werte aus Tabelle $\ref{tab:EvakuierungskurveDreh}$.
\begin{figure}[H]
  \centering
  \includegraphics[scale=0.28]{bilder/evakuierung Dreh.png}
  \caption{Evakuierungskurve Drehschieberpumpe.}
  \label{evakuierungdreh}
\end{figure}

\subsubsection{Leckratenmessung}

\begin{table}[H]
\centering
\label{tab:leck_Dreh1}
\begin{tabular}{c|c|c|c|c}
	{$p \:/\: \si{mbar}$} & {$t_1 \:/\: \si{s} $} & {$t_2 \:/\: \si{s} $} & {$t_3 \:/\: \si{s} $} & {$\bar{t} \:/\: \si{s}$}\\
\midrule
$1,0 \pm \, 0,1$ &0 &0 &0 &0\\
$2,0 \pm \, 0,2$ &   8,53 &  8,91 &  8,75 & $8,73 \pm 0,09$\\
$4,0 \pm \, 0,4$ &   24,89  &  22,93 &  25,06 & $24,29 \pm 0,56 $\\
$6,0 \pm \, 0,6$ &   40,78 &  38,00 &  39,19 & $39,32 \pm 0,66 $\\
$10,0 \pm \, 0,1$ &   73,05 &  70,53 &  71,96 & $71,85 \pm 0,60 $\\
$20,0 \pm \, 0,2$ &   172,71 &  148,99 &  155,35 & $159,02 \pm 5,79 $\\
\end{tabular}
\caption{Gleichgewichtsdruck bei $p_G=(1,0 \pm \, 0,1)  \, \si{mbar}$}
\end{table}

\begin{table}[H]
\centering
\label{tab:leck_Dreh2}
\begin{tabular}{c|c|c|c|c}
	{$p \:/\: \si{mbar}$} & {$t_1 \:/\: \si{s} $} & {$t_2 \:/\: \si{s} $} & {$t_3 \:/\: \si{s} $} & {$\bar{t} \:/\: \si{s}$}\\
\midrule
$0,80 \pm \, 0,08$ &0 &0 &0 &0\\
$1,0 \pm \, 0,1$ &   2,34 &  1,86 &  1,20 & $1,80 \pm 0,27$\\
$2,0 \pm \, 0,2$ &   13,42  &  12,23 &  11,76 & $12,47 \pm 0,40 $\\
$4,0 \pm \, 0,4$ &   32,99 &  31,52 &  32,01 & $32,17 \pm 0,35 $\\
$6,0 \pm \, 0,6$  &   50,15 &  50,65 &  49,51 & $50,10 \pm 0,27 $\\
$10,0 \pm \, 0,1$ &   88,14 &  90,21 &  90,73 & $89,69 \pm 0,65 $\\
\end{tabular}
\caption{Gleichgewichtsdruck bei $p_G=(0,80 \pm \, 0,08) \, \si{mbar}$}
\end{table}

\begin{table}[H]
\centering
\label{tab:leck_Dreh3}
\begin{tabular}{c|c|c|c|c}
	{$p \:/\: \si{mbar}$} & {$t_1 \:/\: \si{s} $} & {$t_2 \:/\: \si{s} $} & {$t_3 \:/\: \si{s} $} & {$\bar{t} \:/\: \si{s}$}\\
\midrule
$0,40 \pm \, 0,04$ &0 &0 &0 &0\\
$0,60 \pm \, 0,06$ &   4,26 &  4,72 &  4,45 & $4,48 \pm 0,11$\\
$1,0 \pm \, 0,1$ &   16,84  &  17,05 &  17,29 & $17,06 \pm 0,11 $\\
$2,0 \pm \, 0,2$ &   45,09 &  43,40 &  44,21 & $44,23 \pm 0,40 $\\
$4,0 \pm \, 0,4$ &   93,98 &  90,09 & 90,21 & $91,43 \pm 1,04 $\\
$6,0 \pm \, 0,6$ &   132,43 &  131,15 &  133,37 & $132,32 \pm 0,53 $\\
\end{tabular}
\caption{Gleichgewichtsdruck bei $p_G=(0,40 \pm \, 0,04) \, \si{mbar}$}
\end{table}

\begin{table}[H]
\centering
\label{tab:leck_Dreh4}
\begin{tabular}{c|c|c|c|c}
	{$p \:/\: \si{mbar}$} & {$t_1 \:/\: \si{s} $} & {$t_2 \:/\: \si{s} $} & {$t_3 \:/\: \si{s} $} & {$\bar{t} \:/\: \si{s}$}\\
\midrule
$0,10 \pm \, 0,01$ &0 &0 &0 &0\\
$0,15 \pm \, 0,02$ &   1,91 &  1,99 &  1,52 & $(5,42 \pm 0,12)$\\
$0,20 \pm \, 0,02$ &   7,70  &  8,64 &  8,67 & $(8,34 \pm 0,26) $\\
$0,30 \pm \, 0,03$ &   18,89 &  21,62 &  19,79 & $(20,10 \pm 0,66) $\\
$0,40 \pm \, 0,04$ &   45,77 &  46,39 & 43,27 & $(45,14 \pm 0,78) $\\
$0,60 \pm \, 0,06$ &   95,86 &  92,23 &  94,75 & $(94,28 \pm 0,88) $\\
\end{tabular}
\caption{Gleichgewichtsdruck bei $p_G=(0,10 \pm \, 0,01) \, \si{mbar}$}
\end{table}

\begin{figure}[H]
  \centering
  \includegraphics[scale=0.43]{bilder/leckdrehfit1.png}
  \caption{Leckrate Drehschieberpumpe 1.}
  \label{leckdreh1}
\end{figure}
\begin{figure}[H]
  \centering
  \includegraphics[scale=0.33]{bilder/leckdrehfit2.png}
  \caption{Leckrate Drehschieberpumpe 2.}
  \label{leckdreh2}
\end{figure}
\begin{figure}[H]
  \centering
  \includegraphics[scale=0.36]{bilder/leckdrehfit3.png}
  \caption{Leckrate Drehschieberpumpe 3.}
  \label{leckdreh3}
\end{figure}
\begin{figure}[H]
  \centering
  \includegraphics[scale=0.36]{bilder/leckdrehfit4.png}
  \caption{Leckrate Drehschieberpumpe 4.}
  \label{leckdreh4}
\end{figure}
