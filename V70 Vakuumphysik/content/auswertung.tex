\section{Auswertung}
\subsection{Bestimmung des Volumens}
Für die folgende Auswertung ist es unumgänglich, das Volumen des Messaufbaus zu bestimmen.
Tabelle $\ref{tab:Bauteile}$ zeigt, aufgelistet nach Pumpen- und Messart die verwendeten Bauteile und ihre Volumina.
Es wird zwischen der Evakuierungsmessung mit der Turbopumpe TE, derselben Messart mit der Drehschieberpumpe DE und der Leckratenmessung
mit der Drehschieberpumpe DL und mit der Turbopumpe TL unterschieden.\newline
Ein Verbindungsstück zur Drehschieberpumpe wurde händisch vermessen.
Das Stück wurde als Zylinder angenähert, sodass sich sein Volumen nach
\begin{align}
  V_{\text{Zy}}=\pi l r^2 &, \qquad \sigma=\sqrt{4 \pi^{2} l^{2} r^{2} \sigma_{r}^{2}  + \pi^{2} r^{4} \sigma_{l}^{2} }\\
\label{eq:VolumenZylinder}
\end{align}
ergibt.\newline
Mit $l=\SI{61 \pm 1}{mm}$ und $r=\SI{6,00 \pm 0,025}{mm}$ wird $V_\text{Zy}=(6{,}9 \pm 0{,}1)\cdot10^{-3}\,l$.
\begin{table}[!hht]
	\centering
	\begin{tabular}{|c|c|c|c|c|c|c|}
		\hline
		Bauteil Nr. & Bauteilbezeichnung & {$\symup{Volumen} \:/\: l$}& TE & TL & DE & DL\\ \hline
		1	&	Tank & $9,5 \pm 0,8$ & 1 & 1 & 1 & 1 \\ \hline
		2	&	langer Schlauch & $0,8 \pm 0,1$ &  &  & 1 & 1 \\	\hline
		3	&	kurzer Schlauch & $0,087 \pm 0,011$ &  &  & 1 & 1 \\	\hline
		4a	&	T-Stück, klein & $0,013 \pm 0,002$ &  &  & 1 & 1 \\	\hline
		4b	&	T-Stück, groß & $0,25 \pm 0,01$ & 1 & 1 & 1 & 1 \\	\hline
		5a	&	Kreuzstück, klein & $0,016 \pm 0,002$ &  &  & 1 & 1 \\	\hline
		5b	&	Kreuzstück, groß & $0,177 \pm 0,09$ & 1 & 1 & 1 & 1 \\	\hline
		6a	&	Handventil 1, offen & $0,015 \pm 0,002$ &  & 1 & 1 & 2 \\	\hline
		6b	&	Handventil 1, geschlossen & $0,005 \pm 0,001$ &  2 & 1 & 2 & 1 \\	\hline
		7a	&	Handventil 2, offen & $0,025 \pm 0,005$ &  &  & 1 &  \\	\hline
		7b	&	Handventil 2, geschlossen & $0,0125 \pm 0,0025$ &  &  &  & 1 \\	\hline
		8a	&	Klappenventil, offen & $0,044 \pm 0,004$ & 1 &  &  &  \\	\hline
		8b	&	Klappenventil, geschlossen & $0,022 \pm 0,002$ &  & 1 & 1 & 1 \\	\hline
		9	&	Querschnittsverengung & $0,067 \pm 0,004$ & 1 &  &  &  \\	\hline
		10	&	Verbindungsstück & $4,72 \pm 0,06$ &  &  & 1 & 1 \\	\hline
	\end{tabular}
	\caption{Auflistung der Bauteile.\cite{anleitung}}
	\label{tab:Bauteile}
\end{table}
Die Teilvolumina der jeweiligen Aufbauten werden addiert und im Folgenden angegeben. Die Fehlerberechnung erfolgt mit Formel ALLGEMEIN!!!.
