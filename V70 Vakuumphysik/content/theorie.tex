\section{Einleitung}

\section{Theorie}
\subsection{Vakuum}
Hingegen der allgemeinen Auffassung, dass Vakuum ein Volumen gänzlich ohne Materie beschreibt, wird in der Physik und Technik
der Vakuumbegriff in Bereiche eingeteilt.
Abhängig vom Druck, der mittleren freien Weglänge und des Strömungsverhaltens lassen sich, wie Tabelle $\ref{tab:bereiche}$ zeigt, vier Bereiche unterscheiden.

\begin{table}[!hht]
\begin{tabular}{c c c c}
 Bereich & Druck $\:/\: mbar$ & freie Weglänge [m] & Strömungsmechanismus \\
 \midrule
 Grobvakuum & 1000 - 1 & $10^{-7}$ - $10^{-4}$ & viskos \\
 Feinvakuum & 1 - $10^{-5}$ & $10^{-4}$ - $10^{-1}$ & Knudsen \\
 Hochvakuum & $10^{-3}$ - $10^{-7}$ & $10^{-1}$ - $10^3$ & molekular \\
 Ultrahochvakuum & $< 10^{-7}$ & $> 10^3$ & molekular \\
\end{tabular}
\caption{Bereichseinteilung des Vakuums}
\label{tab:bereiche}
\end{table}

Die mittlere freie Weglänge beschreibt die durchschnittliche Wegstrecke, die ein Teilchen zurücklegt, ohne mit anderen Teilchen in Wechselwirkung zu treten.
Eben für Gase und der Annahme, dass eine Maxwellsche Geschwindigkeitsverteilung vorliegt gilt die Gasgleichung
\begin{equation}
  \lambda = \frac{k_B T}{\sqrt{2}\pi d^2p}.
\end{equation}
Hierbei ist $k_B$ die Boltzmann-Konstante, $T$ die Temperatur, $d$ der Durchmesser des Moleküls und $p$ der Druck.
\subsection{Strömungsarten}
Für die Charakterisierung einer vorliegenden Strömung und für die Wahl der richtigen Pumpe wird die Knudsen-Zahl $K_n$ als einheitenlose Größe verstanden.
Sie ist definiert durch den Quotienten der mittleren freien Weglänge $\lambda$ und der charakteristischen Länge der Strömung $l_c$
\begin{equation}
  K_n=\frac{\lambda}{l_c}.
\end{equation}
Bei einer Knudsen-Zahl $K_N < 0,01$ wird von einer Kontinuumsströmung und Grobvakuum gesprochen. Hierbei kommt es vermehrt zum Zusammenstoßen der Teilchen im Gas
untereinander. Die mittlere freie Weglänge ist kleiner als die Abmessung der Strömungskanals.
Es wird außerdem in laminarer und turbolenter Strömung unterschieden.
Die laminare Strömung beschreibt Strömungen in Schichten. Die Gasteilchen bleiben immer parallel zueinander.
Nimmt die Strömungsgeschwindigkeit aber zu, so lösen sich die Schichten auf und die Strömung wird turbolent.
Beschrieben wird der Grenzübergang durch die Reynoldszahl
\begin{equation}
  R=\frac{\rho v l_c}{\eta}.
\end{equation}
$\rho$ ist die Dichte, $v$ ist die Strömungsgeschwindigkeit und $eta$ die dynamische Viskosität.
Turbolente Strömungen kommen zum Beispiel beim Abpumpen von Atmosphärendruck.
In der Pumptechnik wird versucht, diese Strömungsart zu verhindern, da durch auftretende Strömungswiderstände Pumpen mit erhöhter Saugkraft benötigt werden.\newline
Für $ 0,01 \leq K_n \geq 0,5$ wird der Begriff der Knudsen-Strömung gebraucht.
Charakteristisch ist diese für den Feinvakuumbereich, der in technischen Anwendungen häufig eine Rolle spielt.\newline
Liegt die Knudsen-Zahl oberhalb von 0,5 so lassen sich kaum noch Wechselwirkungen der Gasteilchen untereinander feststellen.
Es kommt zur molekularen Strömung, bei der die mittlere freie Weglänge sehr viel größer als die Abmessung des Strömungskanals ist.
\subsection{Ideales Gas}
In der Thermodynamik wird das Modell des idealen Gases als idealisierte Vereinfachung für die Beschreibung von Gasprozessen verwendet.
Wechselwirkungen treten ausschließlich durch elastische Stöße der Teilchen untereinander und durch Stöße mit der Wand auf.
Das ideale Gas ist vom Druck $p$, dem Volumen $V$, der Teilchenzahl $N$ und der Temperatur $T$ abhängig und die Zustandsgleichung
folgt zu
\begin{equation}
pV=N k_B T.
\end{equation}
Nach dem Gesetz von Boyle-Mariotte ist der Druck unter $T=const$ antiproportial zum Volumen.
Diese Aussage der Zustandsgleichung des idealen Gases ist von Bedeutung für den später beschriebenen Versuch.
\subsection{Vakuumtechnik}
In der Technik sind Vakuumpumpen unabdingbar.
Je nach Einsatzbereich und Vakuumbereich werden unterschiedliche Pumparten genutzt.
Im vorliegenden Versuch werden zwei Pumparten verwendet.\\
\subsubsection{Drehschieberpumpe}
Eine schematische Darstellung der Drehschieberpumpe ist in Abb.\ref{drehschema} dargestellt. Im Betrieb erzeugt der Rotor zusammen mit den Schiebern einen expandierenden Raum. Durch den so entstehenden Unterdruck wird
das Gas in Arbeitsraum gesogen, bis der zweite Schieber das Einlassventil verschließt. Nun wird durch die Rotation der Raum wieder verkleinert und das Gas muss durch das Auslassventil entweichen.
\begin{figure}[H]
  \centering
  \includegraphics[scale=0.4]{Bilder/drehschieber.png}
  \caption{Schematische Darstellung des Aufbaus einer Drehschieberpumpe.\cite{schemadreh}}
  \label{drehschema}
\end{figure}

\textbf{Turbomolekularpumpe}\\
BESCHREIBUNG DER PUMPEN
