\section{Diskussion}
Die Werte, welche für das Saugvermögen der Turbomolekularpumpe ermittelt wurden, weisen eine starke Abweichung von der Herstellerangabe von $77$\si{\liter}/\si{\second} auf. Dies lässt
sich unter anderem dadurch erklären, dass die Verbindung der Pumpe mit dem Rezipienten über ein Rohr erfolgt, welches einen kleineren Durchmesser hat, wodurch der Strömungswiderstand erhöht wird.
Des Weiteren ist zu bedenken, dass der genutzte Rezipient ein zu geringes Volumen für die genutzte Pumpleistung besitzt. Dies Verursacht eine starke Wechselwirkung der Gasmoleküle mit der Oberfläche des Rezipienten. Eine zusätzliche Fehlerquelle liegt darin, dass
auch bei gründlicher Erhitzung des Rezipienten stets Desorptionseffekte eintreten.\\
Bei der Drehschieberpumpe hingegen stimmen die Messergebnisse innerhalb der Messungenauigkeit mit der Herstellerangabe ($1,1$\si{\liter}/\si{\second}) überein.
