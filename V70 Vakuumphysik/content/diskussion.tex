\section{Diskussion}
Im Generellen zeigen die linearen Plots \ref{turboplot} und \ref{evakuierungdreh} und die aus den Regressionen ermittelten Saugvermögen,
dass dieses druckabhängig ist.
Im Vergleich zu den gegebenen Herstellerangaben von $\SI{77}{l/s}$ weist die Turbomolekularpumpe auf jedem Druckintervall eine deutliche Abweichung auf.
Dies lässt sich unter anderem dadurch erklären, dass die Verbindung der Pumpe mit dem Rezipienten über ein Rohr erfolgt, welches einen kleineren Durchmesser hat, wodurch der Strömungswiderstand erhöht wird.
Des Weiteren ist zu bedenken, dass der genutzte Rezipient ein zu geringes Volumen für die genutzte Pumpleistung besitzt. Dies verursacht eine starke Wechselwirkung der Gasmoleküle mit der Oberfläche des Rezipienten.
Eine zusätzliche Fehlerquelle liegt darin, dass auch bei gründlicher Erhitzung des Rezipienten stets Desorptionseffekte eintreten.
Auch beim händischen Schließen der Ventile kann nicht garantiert werden, dass dies zeitgleich passiert ist.
Bei der Drehschieberpumpe hingegen stimmen die Messergebnisse innerhalb der Messungenauigkeit mit der Herstellerangabe von $\SI{1,1}{l/s}$ überein.
Das Ablesen der Messgeräte und das Stoppen erfolgte fast zeitgleich, sodass schon während der Messung große Abweichungen aufgefallen sind.
Eine Automatisierung der Zeitnahme und ein gesichertes genaues Ablesen der Skala, gerade auf der logarithmischen, würde die gesamten Messungen präzisieren.
Ebenso wäre es förderlich, keine Verengung der Rohre einzubauen, da dies bereits in kleinem Maße große Auswirkungen auf das Saugvermögen hat.\\
Bei der Untersuchung der Drehschieberpumpe per Leckratenmessung fallen für $p_G=(0,10 \pm \, 0,03) \, \si{mbar}$ starke Abweichungen auf, welche durch virtuelles Pumpen enstanden sind. Diese Messwerte wurden für
die Ausgleichsrechnung ausgelassen.
